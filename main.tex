\documentclass{article}
\usepackage[utf8]{inputenc}
\usepackage{graphicx}
\usepackage{hyperref}
\hypersetup{
    colorlinks=true,
    linkcolor=blue,
    filecolor=magenta,      
    urlcolor=cyan,
    pdftitle={CSSU Constitution},
    pdfpagemode=FullScreen,
}
    
\begin{document}

\title{Constitution of the Union}
\author{University of Toronto Computer Science Student's Union}
\date{Revised: February 2022}
\maketitle
\includegraphics[width=\textwidth]{cssu-logo.PNG}
\newpage
\section*{Table of Contents}
\hyperref[sec:1]{1 Introduction} \\
\indent \hyperref[sec:1.1]{1.1 Preamble} \\
\indent \hyperref[sec:1.2]{1.2 Membership} \\
\indent \hyperref[sec:1.3]{1.3 Disputes} \\
\hyperref[sec:2]{2 Council Positions} \\
\indent \hyperref[sec:2.1]{2.1 General Duties and Requirements} \\
\indent \hyperref[sec:2.2]{2.2 Executive Council} \\
\indent \indent \hyperref[sec:2.2.1]{2.2.1 President} \\
\indent \indent \hyperref[sec:2.2.2]{2.2.2 Vice-President} \\
\indent \indent \hyperref[sec:2.2.3]{2.2.3 Treasurer} \\
\indent \indent \hyperref[sec:2.2.4]{2.2.4 Director of Social Events} \\
\indent \indent \hyperref[sec:2.2.5]{2.2.5 Director of Academic Events} \\
\indent \indent \hyperref[sec:2.2.6]{2.2.6 Director of Internal Relations} \\
\indent \indent \hyperref[sec:2.2.7]{2.2.7 Director of External Relations} \\
\indent \hyperref[sec:2.3]{2.3 General Council} \\
\indent \indent \hyperref[sec:2.3.1]{2.3.1 First Year Liaison} \\
\indent \indent \hyperref[sec:2.3.2]{2.3.2 Secretary} \\
\indent \indent \hyperref[sec:2.3.3]{2.3.3 General CSSU Volunteers} \\
\indent \indent \hyperref[sec:2.3.4]{2.3.4 International Student Liaison} \\
\indent \hyperref[sec:2.4]{2.4 Additional Positions} \\
\indent \indent \hyperref[sec:2.4.1]{2.4.1 First Year Representatives} \\
\hyperref[sec:3]{3 Elections} \\
\indent \hyperref[sec:3.1]{3.1 Eligibility} \\
\indent \indent \hyperref[sec:3.1.1]{3.1.1 Voter Eligibility} \\
\indent \indent \hyperref[sec:3.1.2]{3.1.2 Candidate Eligibility} \\
\indent \hyperref[sec:3.2]{3.2 Election Classes} \\
\indent \indent \hyperref[sec:3.2.1]{3.2.1 Annual Election} \\
\indent \indent \hyperref[sec:3.2.2]{3.2.2 By-Election} \\
\indent \hyperref[sec:3.3]{3.3 Chief Returning Officer (CRO)} \\
\indent \hyperref[sec:3.4]{3.4 Calling of Election} \\
\indent \hyperref[sec:3.5]{3.5 Nomination Period} \\
\indent \hyperref[sec:3.6]{3.6 Campaign Period} \\
\indent \hyperref[sec:3.7]{3.7 Voting Period} \\
\indent \hyperref[sec:3.8]{3.8 Resolution of Missing Positions} \\
\indent \hyperref[sec:3.9]{3.9 Advance Voting} \\
\indent \hyperref[sec:3.10]{3.10 Ballots} \\
\indent \indent \hyperref[sec:3.10.1]{3.10.1 Collecting Votes} \\
\indent \indent \hyperref[sec:3.10.2]{3.10.2 Tabulation} \\
\indent \indent \hyperref[sec:3.10.3]{3.10.3 Recount} \\
\indent \indent \hyperref[sec:3.10.4]{3.10.4 Finality of Results} \\
\indent \indent \hyperref[sec:3.10.5]{3.10.5 Transfer of Authority} \\
\indent \indent \hyperref[sec:3.10.6]{3.10.6 Scrutinization of Process} \\
\hyperref[sec:4]{4 Impeachment and Replacement} \\
\indent \hyperref[sec:4.1]{4.1 Resignation} \\
\indent \hyperref[sec:4.2]{4.2 Just Cause for Impeachment} \\
\indent \hyperref[sec:4.3]{4.3 Impeachment Proceedings} \\
\indent \hyperref[sec:4.4]{4.4 Replacements} \\  
\indent \indent \hyperref[sec:4.4.1]{4.4.1 Executive Council Replacements} \\
\indent \indent \hyperref[sec:4.4.2]{4.4.2 General Council Replacements}  \\
\indent \indent \hyperref[sec:4.4.3]{4.4.3 Interim Positions}  \\
\hyperref[sec:5]{5 Miscellaneous}  \\
\indent \hyperref[sec:5.1]{5.1 General Membership Meeting}  \\
\indent \indent \hyperref[sec:5.1.1]{5.1.1 Notice}  \\
\indent \indent \hyperref[sec:5.1.2]{5.1.2 Protocol}  \\
\indent \indent \hyperref[sec:5.1.3]{5.1.3 Quorum}  \\
\indent \indent \hyperref[sec:5.1.4]{5.1.4 Motions}  \\
\indent \hyperref[sec:5.2]{5.2 Amendments}  \\
\indent \hyperref[sec:5.3]{5.3 Official Language of Communication}  \\
\indent \hyperref[sec:5.4]{5.4 Exclusion Rule}  \\
\indent \hyperref[sec:5.5]{5.5 Term of Office}  \\
\hyperref[sec:6]{6 Meta}  \\
\indent \hyperref[sec:6.1]{6.1 Publishing}  \\
\indent \hyperref[sec:6.2]{6.2 License}  \\

\newpage
\section{Introduction} \label{sec:1}
\subsection{Preamble } \label{sec:1.1}
This document will serve as an agreement between undergraduate students in the Computer Science community at the University of Toronto St. George Campus, and the Computer Science Student Union (CSSU). It is the goal of this document to define the organizational structure and processes used to maintain order within the CSSU.
\subsection{Membership } \label{sec:1.2}
Membership in the Computer Science Student Union is granted to any undergraduate student that is enrolled in a Computer Science or Data Science Program of Study, or in a CSC-labeled course in the Faculty of Arts and Science (ie. on the St. George Campus), provided said students are also:
\begin{itemize}
    \item Full-time or part-time undergraduate students who pay the Arts \& Science Student Union (ASSU) fee, where "full-time" and "part-time" are defined by the current \href{https://fas.calendar.utoronto.ca/}{Faculty of Arts \& Science Calendar}.
\end{itemize}

\subsection{Disputes }  \label{sec:1.3}
Any disputes regarding the interpretation of this document will be resolved by the ASSU. The ASSU’s decision will be considered final in all matters.

\section{Council Positions} \label{sec:2}
\subsection{General Duties and Requirements }  \label{sec:2.1}
Each position on the Executive and General councils will require at least 2 hours served in the CSSU office per week. Council members will also be expected to attend regular Council meetings or provide the President with at least 24 hours advance notice of absence.
Each Council member must be a CSSU member as defined in 1.2 "Membership". Each Council member must be enrolled in the Faculty of Arts and Science during both Fall and Spring semesters of their term in office.
Each executive member must have access to email and check it regularly. Each Council member must have access to the primary method of communication for internal correspondence as determined by the Executive Council for the current session.
\subsection{Executive Council }  \label{sec:2.2}
The Executive Council shall be comprised of the President, Vice-President, Treasurer, Director of Social Events, Director of Academic Events, Director of Internal Relations, and Director of External Relations. Only Executive Council members will have the authority to sign on behalf of the union. Any decision on matters of expenditure or property of the union must be approved by a two-thirds majority of the Executive Council, except where otherwise specified by this document.
All Executive Council positions are chosen by an annual election.
\subsubsection{President }  \label{sec:2.2.1}
\begin{itemize}
    \item The President will be responsible for the governance of the union as a whole. The President shall have authority over all space and materials assigned or donated to the union. 
    \item The President shall have the authority to assign a reasonable set of tasks to any Council member. The President will be responsible for calling Council meetings, and for selecting a Council member to chair them. 
    \item The President will, along with the Vice-President and/or the Treasurer, attend all required meetings of the ASSU. 
    \item The President will be responsible for organizing the nomination and distribution of the annual CSSU Teaching Awards.
\end{itemize}
\subsubsection{Vice-President}  \label{sec:2.2.2}
\begin{itemize}
    \item The Vice-President will assist the President in administrative matters.
    \item The Vice-President will supervise and monitor implementation of Presidential directives.
\end{itemize}
\subsubsection{Treasurer} \label{sec:2.2.3}
\begin{itemize}
    \item The Treasurer will be responsible for the keeping of the union’s funds.
    \item The Treasurer will be responsible for dispersing and receiving of funds.
    \item The Treasurer must keep a record of all transactions, especially the retention of physical
    receipts on behalf of the organization.
    \item The Treasurer shall, upon request, provide details regarding the current financial state of the
    union to the President.
    \item The Treasurer will be responsible for preparing and presenting all budget documents.
    \item The Treasurer will provide access to a regularly updated document outlining the current
    financial state of the union.
\end{itemize}

\subsubsection{Director of Social Events} \label{sec:2.2.4}
\begin{itemize}
    \item The Director of Social Events will be responsible for the formulation, preparation, and execution
of social (ie. non-academic) events.
    \item The Director of Social Events will be responsible for organizing a minimum of 3 events in the fall
semester and 3 events in the winter semester. This includes advertising on all CSSU advertising
channels.
\end{itemize}

\subsubsection{Director of Academic Events} \label{sec:2.2.5}
\begin{itemize}
    \item The Director of Academic Events will be responsible for the formulation, preparation, and
execution of academic events.
    \item The Director of Academic Events will be responsible for organizing a minimum of 3 events in the
fall semester and 3 events in the winter semester. This includes advertising on all CSSU
advertising channels.
\end{itemize}

\subsubsection{Director of Internal Relations} \label{sec:2.2.6}
\begin{itemize}
    \item The Director of Internal Relations will be responsible for the interactions and relationships
between the union and all other organizations within the University. This is including but not
limited to the Department of Computer Science, student clubs and research groups. This
applies across all three campuses at the University.
\end{itemize}

\subsubsection{Director of External Relations} \label{sec:2.2.7}
\begin{itemize}
    \item The Director of External Relations will be responsible for the interactions and relationships
between the union and all organizations outside of the University. This is including but not
limited to sponsors, recruiters and service providers.
\end{itemize}

\subsection{General Council} \label{sec:2.3}
The General Council consists of all CSSU Council members that were not elected. Once
elected, the Executive Council must announce the availability of General Council positions at
least via the official communication channels, and solicit applications from the CSSU
membership. Any applicant may be appointed to any General Council position provided that the
majority of the Executive Council agrees to the appointment.
The Council must consist of at least one member in each of the following years as determined
by the Faculty of Arts and Science: 1st, 2nd, 3rd, and 4th or greater.
\subsubsection{First Year Liaison} \label{sec:2.3.1}
Duties:
\begin{itemize}
    \item Advise first year students and aid their transition to university life
    \item Plan and promote a minimum of one social event per semester targeted towards first
year students
    \item Plan and promote an informative seminar early in the academic year targeted toward
first year students
    \item Hold a minimum of one weekly office hour for the purpose of addressing the concerns of
first year students, in addition to the regular office hours required
    \item Coordinate with First Year Representatives to ensure first year student engagement in
all events and activities carried out by the union.
\end{itemize}
Skills & Requirements:
\begin{itemize}
    \item Ability to establish trust and credibility with first year students
    \item Have at least 1.0 FCEs at the beginning of the academic year
\end{itemize}

\subsubsection{Secretary} \label{sec:2.3.2}
Duties:
\begin{itemize}
    \item Keep track of CSC-course scheduling so events can be organized accordingly
    \item Attend all Council meetings and take minutes
    \item Maintain a list of internal and external contacts for use by the current and future councils
    \item Maintain a calendar of internal union events and meetings that is accessible to all
council members
    \item Maintain a list of volunteers from the general membership
\end{itemize}
Skills & Requirements:
\begin{itemize}
    \item Strong organizational skills
\end{itemize}

\subsubsection{General CSSU Volunteers} \label{sec:2.3.3}
Duties:
\begin{itemize}
    \item Help any appointed position member with tasks, especially around bigger events
\end{itemize}

Skills & Requirements:
\begin{itemize}
    \item Organized
    \item Responsible
    \item Willing to work in different positions
\end{itemize}
This position allows member to be part of the CSSU and help with numerous different tasks, as
well as help specific members with larger tasks. In the event that an appointed CSSU member is
not performing their tasks sufficiently, and a General member performs them better, the General
Member can replace the appointed member at the discretion of a majority vote from the
Executive Council. This can be used to ensure that the total number of council members is odd,
avoiding tie votes.
\subsubsection{International Student Liaison} \label{sec:2.3.4}
Duties:
\begin{itemize}
    \item Ensure events and spaces are accessible and welcoming for international students
    \item Provide insight and suggestions for other council members as they organize events
\end{itemize}
Skills & Requirements:
\begin{itemize}
    \item Existing relationship with the international student community
    \item Understanding of the challenges faced by international students
\end{itemize}

\subsection{Additional Positions} \label{sec:2.4}
These positions are required to be filled but are not included in the council.
\subsubsection{First Year Representatives} \label{sec:2.4.1}
Duties:
\begin{itemize}
    \item Engage first year students, and inform them of CSSU and its upcoming events
    \item Make announcements about the CSSU itself and its events before the start of class
    \item Promote events to first year student body
\end{itemize}
Skills & Requirements:
\begin{itemize}
    \item Strong public speaking abilities
    \item Strong verbal communication skills
    \item Strong social skills; be able to talk to almost anyone at anytime
    \item Have at most 2.0 FCEs in CSC code courses by the end of the academic year
\end{itemize}
Please note that the First Year Representative positions will not be filled out until the beginning of the academic year, when we have a new incoming group of freshmen. There will be a minimum of three First Year Representative positions. First year reps are eligible to hold other positions within the CSSU simultaneously.

\section{Elections} \label{sec:3}
\subsection{Eligibility} \label{sec:3.1}
\subsubsection{Voter Eligibility} \label{sec:3.1.1}
Voting privileges shall be awarded to all CSSU members as defined in 1.2, “Membership”. Each eligible voter will receive exactly one vote per issue.
\subsubsection{Candidate Eligibility}  \label{sec:3.1.2}
Any candidate in any election must meet the requirements specified in 1.2, “Membership”.
\subsection{Election Classes} \label{sec:3.2}
\subsubsection{Annual Election}  \label{sec:3.2.1}
An annual election must be held near the end of every Spring semester to determine replacements for the upcoming academic year. The election date will be set at the discretion of the current President, however it must be within four weeks of the conclusion of the Spring semester. The election may not be held during the final week of classes.
\subsubsection{By-Election} \label{sec:3.2.2}
A by-election will be held when necessary to determine a candidate for a vacant position. The date will be set by the CRO who will be appointed by two-thirds vote of the Council.
\subsection{Chief Returning Officer (CRO)}  \label{sec:3.3} 
The CRO shall be nominated by the President and appointed by unanimous approval of the existing Executive Council to oversee election proceedings. The CRO shall, at their discretion, with the authority of the President, appoint Council members to aid in the elections. Neither the CRO nor any appointed aides may be candidates in the upcoming election.
The CRO will have discretionary powers to interpret and execute the defined election protocols as outlined in chapter 3 of this document.
The CRO will have the authority to disqualify any candidate disobeying election protocol. Once appointed, the CRO may only be replaced after a formal request issued by two-thirds of the current Executive Council members.
\subsection{Calling of Election}  \label{sec:3.4}
Upon deciding on an appropriate date and successful nomination of a CRO, the President shall publicly declare the election process to have begun. The President should ensure through reasonable means that the community is aware of when and where the election itself is occurring, and confirm the identity of the CRO. This includes, at a minimum, notification through the CSSU external communications channels. The election must take place no earlier than two weeks after the announcement to the official communication channels. In the case of a presidential by-election, the Interim President will perform the aforementioned duties.
\subsection{Nomination Period} \label{sec:3.5}
To become a candidate for an elected position, one must provide a government issued photo ID
or Tcard. The CRO may determine the exact format in which these are provided and must retain
copies of them for at least and no longer than 3 business days following election day.
Nominations may only be made by eligible voters, as laid out in 3.1.1, “Voter Eligibility”. Details
of this process shall be determined by the CRO and made publicly available in advance of the
nomination period. The CRO may not nominate any candidate.
The nomination period must last at least 5 complete business days. During the nomination
period, no candidate may engage in campaigning.
In the event that only exactly one individual is nominated for a position, excluding Executive
Council positions, they do not run in an election, but immediately win the position. They assume their position after the election as specified in 3.10.4 “Transfer of Authority”. In
the case of an uncontested Executive Council nomination a yes/no vote will still be held.
\subsection{Campaign Period} \label{sec:3.6}
After the nomination period has ended, the candidates may campaign up to and excluding the
Voting Period.
Campaign rules will be determined by the CRO and made known in advance of the campaign
period. The campaign period must last at least 5 complete business days.
\subsection{Voting Period} \label{sec:3.7}
The CRO will decide, according to University of Toronto’s public health measures, on the format of the election: online or in-person.\\ 

\textbf{If the election is conducted online:}
The Voting Period will be conducted over a period of multiple days as determined by the
CRO.
The CRO will procure a fair and reliable means of voting online, with a preference to voting
technologies familiar to UofT Students.
Each candidate may choose to submit one advertisement to be displayed on the CSSU
social media. The size and any other constraints of this advertisement will be determined by
the CRO. Candidates shall also submit a written statement to be displayed on the ballot, as
long as the statement complies with all rules specified by the CRO and advertised prior to
the close of the nominations period.
Any volunteers or other parties managing the voting technology must be provided with a list
of advance voters who have already submitted their advance vote. They must refuse to
allow anyone on the list to vote.The CRO is given the discretion to interpret the constitution in accordance with all Provincial
Public Health Guidelines and University policies, as long as such interpretations remain in
the spirit of this constitution. The CRO, at their discretion, will notify all candidates and the
CSSU General Council of such interpretations. Should a CSSU Council member disagree
with this interpretation, they can overrule the CRO by a unanimous vote of General Council
and CSSU executive members present at a combined Executive and General Council
meeting.\\

\textbf{If the election is conducted in-person:}
The Voting Period must begin and end within the same business day, and must last at least 6
hours.
The CRO will select a Voting Area, which must be reserved exclusively for election matters
during the Voting Period.
Each candidate may choose to submit one 8.5” × 11” advertisement to be displayed within the
Voting Area, placed at the CRO’s discretion. There must be a secluded area within the Voting
Area for each voter to privately complete their ballot.
No candidate may attempt to communicate with voters entering or within the Voting Area. A
candidate may enter the Voting Area only to complete their ballot, and must leave
immediately afterwards. At the CRO’s discretion, a candidate may be barred from the Voting
Area for the duration of the Voting Period.
Any volunteers managing the voting area must be provided with a list of advance voters who
have already submitted their advance vote. They must refuse to allow anyone on the list to vote.
\subsection{Resolution of Missing Positions} \label{sec:3.8}
In the event that nobody is nominated for a position, or in the event that a tie results from the
final tally of votes, a by-election must be held for the empty position (and any other empty
positions) either immediately, if time permits as judged by the CRO, or else during the first
month of the next school year (in the Fall semester).
\subsection{Advance Voting} \label{sec:3.9}
Any eligible voter may choose to vote in advance of the voting period. The CRO must provide
Advance Voting forms online and advertise them on the CSSU external communication
channels before the nomination period begins. Instructions on how to fill out the ballots must
also be provided.A voter wishing to cast an advance ballot must provide a completed Advance Voting form to the
CRO no less than 48 hours prior to the Voting Period, at which point they will be given a
ballot.
The voter must return the ballot to the CRO before the start of the Voting Period, folded inside a
sealed completely unmarked envelope. Any breach of this protocol may result in the ballot being
declared spoiled, at the discretion of the CRO. The CRO should make available their contact
information for the purpose of coordinating these transactions.
Once the envelope is handed to the CRO, the voter is marked down as an advance voter, and
may not amend their vote, and may not vote in person.
The CRO must keep all advance ballots in their possession until tabulation takes place. No
advance ballot envelope may be opened until tabulation takes place as outlined in 3.10.1,
“Tabulation”.
The CRO must verify that all persons casting advance ballots are eligible voters as defined in
3.1.1.

\subsection{Ballots} \label{sec:3.10}
Each ballot will contain a list of positions and for each a list of candidates. The candidates will
be ordered from top to bottom, sorted in alphabetical order.
In the event that there is only one candidate for a position, a Yes/No choice must be provided.A
ballot's vote on a position is considered invalid under the following conditions:
● More than one choice for the position is selected, or
● Both “Yes” and “No” are indicated for the position, for a position with only one candidate
running, or
● A marking has been made but it is unclear what choice is being indicated, or
● No mark has been made for the position.
A ballot is not considered spoiled if a vote on a position is invalid. An invalid vote shall be
considered to indicate abstinence from a vote on a particular issue. The number of invalid votes
for each position should be reported along with the tally of votes for candidates.
A vote for a position may be considered invalid at the CRO’s discretion. In addition, an entire
ballot may be considered spoiled at the CRO’s discretion. Spoiled ballots will be discarded from
the counting process entirely.
\subsubsection{Collecting Votes} \label{sec:3.10.1}
Voters must register their names, student numbers, and UTOR IDs at a booth run by CSSU
volunteers not running for any position. The CSSU must verify that voters are CSSU members
to the best of their ability, and must exclude anyone who already submitted an advance ballot.
Alternatively, votes may be collected online as opposed to a physical booth.
\subsubsection{Tabulation} \label{sec:3.10.2}
In the case of a paper voting system:
For each vote, a paper ballot must be placed within a sealed box.
Immediately following the end of the election period, the tabulation period begins. At this time,
the
advance ballot envelopes may be opened, and the ballots inside must be placed inside the
sealed box.
Then the box must be shuffled (i.e. shaken). Only after this may the box be unsealed, and it
must be unsealed under the supervision of the CRO.
The CRO must keep all completed ballots until they are deemed destroyable as per 3.10.4.In the case of an online voting system:
The CRO must ensure that the online votes are only recorded during the designated voting
period. They must also ensure that the link to vote is shared on CSSU communications
channels as well as on a notice in the CSSU office.
Ballot counting must be done under supervision of the CRO. Upon completion, the CRO must
provide the current President with election results immediately. The current President must then
make the results known publicly within 24 hours.
\subsubsection{Recount} \label{sec:3.10.3}
Any candidate may demand one recount if there is less than a 10% margin of victory in their
category.
At least one recount must be performed if there is less than a 5% margin of victory.
The CRO must personally perform each recount.
If the tabulation changes after a recount, an additional recount must occur.
Any current Executive Council member may demand one recount in any category, for any
reason, regardless of victory margin.
\subsubsection{Finality of Results} \label{sec:3.10.4}
After 3 complete business days of election results or the latest recount being made available
publicly, no more recounts may be demanded, and the ballots must be destroyed. At this point
the election results are final and the CRO is dismissed from duty.
\subsubsection{Transfer of Authority} \label{sec:3.10.5}
Following an annual election, the outgoing and incoming Executive Councils must arrange a meeting before the end of the first week of summer courses to transfer ownership of the primary bank account, access to assigned space, and any keys. After control of the primary bank account has been transferred to the incoming Executive Council, they are considered the current Executive Council and assigned to their respective roles. The previous Executive Council is thus dismissed from duty. If the transfer of authority is not made by the end of the first week of summer courses, then the incoming Executive Council is still considered to be the current Executive Council, and must make its best effort to gain control over all CSSU resources. Following a complete by-election the winning candidate is immediately considered instated.
\subsubsection{Scrutinization of Process} \label{sec:3.10.6}
Each candidate may choose to to appoint a Scrutineer to oversee election proceedings on their behalf. The candidate must inform the CRO 24 hours before the election period as to their choice of Scrutineer, and provide contact information. The Scrutineer may be present in the Voting Area and during tabulation and any recounts. Should a Scrutineer observe any error or misconduct which will result in corruption of the democratic process, they must immediately make known the specifics of the complaint to the Arts and Science Students’ Union. Should a Scrutineer be acting in an obstructive manner to the election procedures they may be barred from the Voting Area at the CRO’s discretion. The CRO should immediately inform a member of the current Executive Council, as well as immediately inform the represented candidate that a new Scrutineer will be required. Tabulation must halt until a new Scrutineer is found, or until an hour has passed since the Scrutineer was removed and the candidate is successfully notified. If multiple successive Scrutineers are barred from the Voting Area, then the CRO may, at their discretion, remove the candidate from the election, or remove the candidate's right to a Scrutineer. No candidate in the current election may function as a Scrutineer. Neither the CRO nor any appointed aide to the CRO may serve as a Scrutineer. A single Scrutineer may serve multiple candidates.

\section{Impeachment and Replacement} \label{sec:4}
\subsection{Resignation}  \label{sec:4.1}
Any Council member is free to resign their position at any time. A resignation will be considered official when the President is notified in writing by the resigning party. If the President is resigning, the Vice-President must be notified in writing.
\subsection{Just Cause for Impeachment}  \label{sec:4.2}
Should any Council member fail to meet the duties or requirements of their position, abuse their position for personal reasons, fail to respond to communications within a reasonable time-frame without just cause or otherwise fail to act in good faith, they should be recommended for impeachment by some member of either the community or one of the Councils. Should one Executive Council member agree with the impeachment recommendation, impeachment proceedings will begin.
\subsection{Impeachment Proceedings}  \label{sec:4.3}
The President (or Vice-President, if the President is being considered for impeachment) will notify the individual that they are being considered for impeachment, and explain the reasoning. A meeting will be set between the individual and the Executive Council at a reasonable time during which the individual will be questioned. After the Executive’s questioning is complete, a vote will be taken in private as to whether the proceedings will continue.
Should the individual fail to attend the meeting or refuse to compromise on a meeting schedule, the Executives may vote summarily. If an Executive Council member is being considered for impeachment, they will be allowed to vote in the proceedings. Should at least two-thirds of the Council present at the meeting agree to impeachment, then the individual is considered impeached and dismissed from their position immediately. If the member in question is on the Executive Council then an official announcement must be made on the CSSU communication channels within 24 hours of the vote.
\subsection{Replacements} \label{sec:4.4}
\subsubsection{Executive Council Replacements}  \label{sec:4.4.1}
Executive Council members must be replaced by means of a by-election or by a vote of the general membership at a General Meeting. The procedure for a by-election is laid out in 3.2.2, and the procedure of a General Meeting is as set forth in 5.1, “General Meeting”. If a position on the Executive Council is to be filled by means of a vote of the general membership at a General Meeting, notice of this must accompany the two weeks advance notice of a general meeting outlined in 5.1.1, “Notice”.

In the case of medical or personal emergency, the individual is expected to provide
appropriate notice to the Executive Council no later than 24 hours after the intended meeting date.

\subsubsection{General Council Replacements}  \label{sec:4.4.2}
Should any General Council position become vacant, a replacement will be chosen in the manner described in 2.3.
\subsubsection{Interim Positions}  \label{sec:4.4.3}
If a council position's duties require someone to hold the position immediately, then the President may appoint an interim council member. If there is no President, then the Vice President may do this, and if there is no Vice President nor President, then the Treasurer may do this.
An interim council member has all the duties of the position, but must step down immediately when a replacement is chosen by one of the above processes.

\section{Miscellaneous} \label{sec:5}

\subsection{General Membership Meeting}  \label{sec:5.1}
At least one General Meeting must be held per academic semester, while classes are in session, for each of the Fall and Spring semesters. Any Executive Council member may (unilaterally) call a General Meeting.
\subsubsection{Notice}  \label{sec:5.1.1}
One week advance notice must be given to the student body before the date of a General Meeting. The meeting shall be chaired by the member of the Executive Council who called the meeting.
\subsubsection{Protocol}  \label{sec:5.1.2}
The protocol for General Meetings is as defined in the constitution of the Arts and Science Students’ Union.
\subsubsection{Quorum}  \label{sec:5.1.3}
The quorum at all General Meetings shall be 20 members or 20% of the Council membership, whichever is less.
\subsubsection{Motions}  \label{sec:5.1.4}
Any matter of policy may be put forth to a vote of the general membership at a Council Meeting.
\subsection{Amendments} \label{sec:5.2}
\subsubsection{Compliance with the law}  \label{sec:5.2.1}
At all times, and when it serves the best interests of CSSU members, the CSSU will endeavor to
operate in compliance with all Federal, Provincial, and Municipal laws and guidelines (hereafter "the law") and in accordance with University and Departmental policy.
At any point, should a member of the General Council identify a clause in this constitution which would require the CSSU to violate the law or University policy, a vote will be conducted on an appropriate amendment to the Constitution to bring the CSSU into compliance with the law. Such a vote requires a vote of at least 50\% of the executive council and 50\% of the General Council. The amendment will be in effect as of the date of the successful passing in this vote, but the CSSU Executive will be required to call a General Members Meeting at the soonest possible time after and the amendment will be approved through the normal process in clause 5.2.2
In the case of emergency, the CSSU can write temporary amendments to the CSSU constitution that will remain in effect until a predetermined date, before requiring renewal through the process outlined in this clause.
Nothing in this clause will permit the CSSU to amend this clause of the constitution.

\subsubsection{Changes}  \label{sec:5.2.2}
Changes to this document shall only be permitted by a two thirds supermajority of the general membership in attendance at a General Meeting (as defined in 5.1, “General Meeting”), and according to the regulations of the Arts and Science Students’ Union.
An amendment may only be discussed at a General Meeting if it is announced on the agenda, and with unanimous consent among Executive Council members.
\subsection{Official Language of Communication}  \label{sec:5.3}
The union adopts English, the University of Toronto’s language of instruction, as its official language.
All official and unofficial communications on behalf of the union, including but not limited to its constitution, website, advertisements, notices, and campaign materials on behalf of individuals running in a CSSU election, shall be written entirely and exclusively in English.
\subsection{Exclusion Rule}  \label{sec:5.4}
No candidate may run for more than one position during a given election. No individual may hold more than one Council position simultaneously. An individual may run in a by-election while currently holding a Council position, however they must resign the previously held position if elected. The previously held position will subsequently be filled in accordance with the procedures laid out in 4.4, “Replacements”.
\subsection{Term of Office}  \label{sec:5.5}
Upon the new Executive Council being instated, all previous Council positions are dismissed.

\section{Meta} \label{sec:6}
\subsection{Publishing}  \label{sec:6.1}
This document must be linked to from the CSSU webpage.
\subsection{License}  \label{sec:6.2}
To the extent possible under law, the authors and the CSSU have waived all copyright and related or neighboring rights to this document.
This document is licensed under the Creative Commons 0 (CC0) license. A full copy of the license text is available at: http://creativecommons.org/publicdomain/zero/1.0/
\end{document}
